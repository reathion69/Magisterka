\chapter{Program układania planu zajęć}

Implementacja algorytmu została napisana w języku C\#. Język ten został wybrany z uwagi na dużą liczbę bezpłatnych bibliotek oraz możliwość dostępu do bardzo dobrej dokumentacji technicznej. Nie bez znaczenia był także fakt dużej popularności języka, co skutkuje większymi możliwościami uzyskania pomocy na forach programistycznych.

\section{Użyte narzędzia}

Podczas implementowania algorytmu wyszukiwania z tabu użyto następujących narzędzi:

\begin{itemize}
	\item Język programowania C\# - jest to obiektowy język programowania, którego początki sięgają roku 1998. Został zaprojektowany dla firmy Microsoft przez Andersa Hejlsberga. Program napisany w tym języku jest kompilowany do języka \textit{Common Intermediate Language} (CIL), który jest wykonywany w środowisku \textit{.NET Framework}. Jest to język prosty, o dużych możliwościach i wielu cechach wspólnych z językami programowania C++ oraz Java \cite{Csharp}. 
	
	\item Biblioteka Linq - Language-Integrated Query (LINQ) pozwala odczytywać dane pochodzące z różnych źródeł w postaci obiektów. Biblioteka udostępnia funkcje filtrujące i wyszukujące, które odpowiednio użyte w znacznej mierze przyspieszają działanie programu. W implementacji algorytmu, którego dotyczy praca, zapytania Linq ułatwiają komunikację z bazą danych oraz czytanie dokumentów XML.
	
	\item Windows Forms - interfejs programowania graficznych aplikacji należący do środowiska \textit{.NET Framework}. Służy do tworzenia aplikacji z graficznym interfejsem użytkownika, umożliwiająca obsługę zdarzeń napływających od użytkownika. Obecnie interfejs wypierany jest przez interfejs Windows Presentation Foundation  (\textit{WPF}). Jednak mimo tego, dzięki swej prostocie i przejrzystości, jest dalej używany przez programistów.
	
	\item Entity Framework - jest to bezpłatne oprogramowanie typu Object Relational Mapping (ORM), pozwalające odwzorować relacyjną bazę danych za pomocą architektury obiektowej. Dzięki temu narzędziu programista może w łatwy sposób zaprojektować bazę danych nie mając wiedzy na temat języka SQL. Stosując podejście \textit{code first} można zaprojektować tylko klasy bazowe, a resztę powierzyć narzędziu Entity Framework, które utworzy odpowiednią strukturę bazy danych.
	
	\item Microsoft Visual Studio 2015 - zintegrowane środowisko programistyczne firmy Microsoft, umożliwiające pisanie i kompilowanie aplikacji w językach takich jak C\#, C++, C czy Visual Basic, na różne platformy. Dzięki dużej liczbie narzędzi umożliwia łatwe testowanie i debagowanie kodu, oraz pozwala szybko tworzyć nowe aplikacje. Jest to naturalny wybór dla języka programowania C\#.
	
	\item SQL Server 2014 Managment Studio - zintegrowane środowisko do zarządzania bazami danych. Zawiera narzędzia do konfigurowania i monitorowania baz danych. Umożliwia budowę zapytań i oglądanie tabel w bazie danych. Autor pracy wykorzystał to środowisko do testowania poprawności wykonanych operacji i analizy uzyskanych wyników.
\end{itemize}


\section{Wprowadzanie danych}

Pierwszym problemem, który stanął przed autorem tej pracy, było wprowadź danych, korzystając z formatu XHSTT. Format ten został omówiony w rozdziale czwartym. Ponieważ dane zawarte w tym formacie zapisane są w języku XML, skorzystano z biblioteki systemowej  \textit{Xml.Linq}, która pozwala traktować dokument XML jako bazę danych i wysyłać do niej stosowne zapytania. Przykład wprowadzania instancji problemu układania planu zajęć za pomocą tej biblioteki przedstawiony został na rysunku 6.1. Biblioteka \textit{Xml.Linq} stanowiła spore ułatwienie, jednak w dalszym ciągu trudności sprawiał złożony charakter dokumentu, ponieważ każde zagnieżdżenie musiało być wczytane za pomocą odpowiedniego wiersza kodu. 

\begin{figure}
	\centering
	\includegraphics {linqprzyklad}
	\caption{Przykład wprowadzania instancji problemu układania planu zajęć za pomocą biblioteki \textit{Xml.Linq}.}
	\label{fig: linqprzyklad}
\end{figure}

Wprowadzone dane zapisywane są w bazie danych, dlatego każde kolejne wykonanie aplikacji umożliwia pracę na już wczytanych danych. Jak stwierdzono w poprzednich rozdziałach, założono, że wprowadzone dane zawierają zdarzenia z przypisanymi wcześniej zasobami, takimi jak nauczyciel, sala i grupa.


\section{Specyfikacja wewnętrzna programu}

Poprzedni podrozdział miał na celu przedstawienie pomysłu i sposobu w jaki zaimplementowany został algorytm wyszukiwania z tabu. Ten podrozdział zawiera szczegóły techniczne implementacji, opis użytych narzędzi, używanych klas oraz zawiera najistotniejsze fragmenty kodu.



\subsection{Opis klas i ważniejszych funkcji}

Aplikacja zapisana jest w kilku folderach. Każdy z folderów zawiera klasy realizujące odrębne funkcje aplikacji. Folder \textit{Code} zawiera klasy realizujące algorytm wyszukiwania.  W folderze \textit{Data} znajdują się klasy służące do translacji plików wejściowych XML na obiekty, i zapisywania nowo powstałych obiektów do bazy danych. Folder \textit{Model} zawiera klasy odwzorowujące obiekty znajdujące się w bazie danych. Znajdują się więc tu klasy takie jak \textit{Instance, Event, Resource}. Ostatni folder, \textit{ViewModel}, przechowuje klasy realizujące przygotowanie danych to zaprezentowania ich w interfejsie graficznym. Poza folderami znajduje się osobna klasa \textit{Form1}, która obsługuje interfejs graficzny użytkownika.

Poniżej przedstawiono ważniejsze klasy programu wraz z opisem funkcji w nich zawartych.

\begin{itemize}
	\item  \textit{Code/SolutionManager} - klasa zawiera szkielet algorytmu wyszukiwania z tabu. Jest to główna klasa algorytmu.  Znajdują się w niej funkcja rozwiązująca problem układania planu zajęć dla instancji o podanym identyfikatorze. Klasa ta zawiera również funkcje zapisującą raport działania algorytmu do pliku txt oraz funkcję wyświetlającą rozwiązanie na konsolę. Główna funkcja klasy zaprezentowana została na rysunku 6.2.
	
	\begin{figure}
	\centering
	\includegraphics {ResolveSimpleProblem}
	\caption{Funkcja \textit{ResolveSimpleProblem} zawierająca szkielet algorytmu Tabu search. }
	\label{fig: ResolveSimpleProblem}
	\end{figure}

	\item \textit{Code/TabuSearch} - klasa zawiera statyczne funkcje służące do rozwiązywania problemu za pomocą algorytmu wyszukiwania z tabu. Znajdują się tu funkcję generujące dostępne ruchy, losowe rozwiązanie startowe, aktualizujące listę tabu, generujące sąsiedztwo i wybierające z nich najlepszego osobnika. W tej klasie znajdują się najważniejsze fragmenty kodu implementujące algorytm wyszukiwania z tabu. Funkcja generująca sąsiedztwo i wybierająca z niego najlepszą instancje została zaprezentowana na rysunku 6.3.

	\begin{figure}
	\centering
	\includegraphics {SelectbestInstance}
	\caption{Funkcja \textit{SelectBestInstanceFromNeighborhood} generująca sąsiedztwo i wybierająca najlepszego sąsiada. }
	\label{fig: SelectbestInstance}
	\end{figure}

	\item \textit{Code/EvaluationFunction} -  w skład tej klasy wchodzą funkcje oceniające bieżące rozwiązanie uzyskane przez algorytm. Znajduje się tu implementacja wszystkich ograniczeń, jakie mają być uwzględnione w rozwiązaniu. Klasę tę można w łatwy sposób rozszerzyć dodając kolejne funkcje oceniające, które odpowiadają kolejnym ograniczeniom.
	 
	\item \textit{Code/TabuItem} - klasa reprezentująca pojedynczy ruch w algorytmie wyszukiwania z tabu. Jako ruch rozumiana jest para dwóch indeksów, pierwszy z nich odpowiada identyfikatorowi zdarzenia, a drugi to identyfikator okna czasowego, które w danym ruchu jest do tego zdarzenia przypisane.  Elementy klasy przechowywane są na liście tabu, dlatego klasa zawiera dodatkową właściwość, jaką jest czas trwania na liście tabu dla danego elementu.
	 
	\item  \textit{Data/XMLLoader} - w tej klasie zawarte są funkcje przekształcające dane wejściowe w formacie XHSTT, zapisane w języku XML, na klasy modelu. Znajduje się tu również funkcja zapisująca odczytaną instancje do bazy danych.
	
	\item \textit{Form1} - klasa obsługująca interfejs użytkownika. Zawiera funkcje reagujące na każdą wykonaną przez użytkownika czynność oraz funkcje wypisujące dane do odpowiednich obiektów interfejsu.
\end{itemize}

\section{Specyfikacja zewnętrzna programu}

	Po zainicjowaniu działania aplikacji wyświetlane jest okno główne przedstawione na rysunku 6.4. Przy pierwszym wykonaniu aplikacji użytkownik powinien wczytać instancje z pliku wejściowego. Może to wykonać za pomocą przycisku ,,Load instance from file'', który znajduje się w prawym górnym rogu okna. Otworzy się okno dialogowe, gdzie należy wskazać plik XML zawierający dane wejściowe dla problemu rozwiązywania planu zajęć, zapisane w formacie XHSTT. Po wczytaniu wielu instancji (lub po ponownym zainicjowaniu działania aplikacji), użytkownik może wybrać instancje do rozwiązania. Służy do tego rozwijana lista w lewym górnym rogu okna.
	
	Przycisk ,,Resolve'' służy do wykonania algorytmu. Działanie algorytmu jest monitorowane, a postępy wyświetlane są na konsoli na dole okna aplikacji. Po zakończeniu obliczeń, ułożony plan zajęć można oglądać w tabeli. Rozwijana lista nad tabelą służy do wyboru klasy, dla której ma zostać zaprezentowany plan zajęć.
	
	Użytkownik może zmieniać parametry algorytmu. Algorytm ma trzy parametry, których opis znajduje się poniżej:
	
	\begin{itemize}
		\item Iterations - maksymalna liczba iteracji jaką algorytm wykona w czasie rozwiązywania problemu układania planu zajęć. Jeżeli algorytm wcześniej znajdzie rozwiązanie, na które nie jest nałożona żadna kara, to zakończy swoje działanie.
		\item Tabu duration - parametr oznaczający czas pozostawania pojedynczego ruchu na liście tabu. Długość mierzona jest liczbą iteracji algorytmu.
		\item Neighborhood - rozmiar sąsiedztwa jakie zostanie losowo wybrane i przeszukane podczas jednej iteracji algorytmu. Rozmiar sąsiedztwa nie powinien być większy od liczby wszystkich dostępnych ruchów, które może wykonać algorytm (pełne sąsiedztwo).
	\end{itemize}
	
	\begin{figure}
	\centering
	\includegraphics[width=\textwidth] {Aplikacja}
	\caption{Okno główne aplikacji.}
	\label{fig: Aplikacja}
	\end{figure}