\chapter{Problem układania planu zajęć.}
\section{Ograniczenia planu zajęć.}
Problem układania planu zajęć można sprowadzić do przyporządkowania określonym wydarzeniom odpowiednich okien czasowych i zasobów, tak by spełnić początkowe założenia. Dla planu zajęć wydarzeniami będą poszczególne lekcje odbywające się w ciągu jednego tygodnia, dostępnymi oknami czasowymi będą godziny w których mogą odbywać się te zajęcia (np. od poniedziałku do piątku w godzinach między 8:00 a 18:00), natomiast zasobami będzie cała reszta danych przypisana do zajęć: grupy uczniów, prowadzący zajęcia, sala.

Układanie planu zajęć wymaga przestrzegania odpowiednich ograniczeń. Nie można w prosty sposób wypełnić okien czasowych według jakiejś przyjętej kolejności ponieważ najprawdopodobniej pojawią się konflikty i plan zajęć nie będzie możliwy do zrealizowania. By tego uniknąć trzeba zachować tzw. ograniczenia twarde, czyli takie, które kategorycznie muszą zostać spełnione by plan mógł być możliwy do zrealizowania. Istnieją również ograniczenia miękkie, które określają jakość planu końcowego. Ograniczenia te nie muszą zostać spełnione, ale jeżeli jest to możliwe, algorytm układający plan zajęć bierze je pod uwagę.

Przykłady ograniczeń:
\begin{itemize}
	\item Twarde:
	\begin{itemize}
		\item Niepodzielność zasobów
		\item Przypisanie wszystkich zajęć do okien czasowych
	\end{itemize}
	\item Miękkie:
	\begin{itemize}
		\item Brak okienek dla studentów
		\item Odpowiednie przerwy między zajęciami
		\item Jak najmniej dni roboczych dla prowadzących
		\item Brak bloków zajęć danego typu
	\end{itemize}
\end{itemize}

Niepodzielność zasobów oznacza, że jeden nauczyciel nie może być w dwóch miejscach jednocześnie, więc może prowadzić tylko jedne zajęcia w danym oknie czasowym, w jednej sali nie mogą się odbywać w tym samym czasie różne zajęcia, a dla danej grupy lekcje nie mają prawa się nakładać. Naruszenie tego ograniczenia było by fizycznie nie możliwe. 
Obowiązek przypisania zajęć do jakichś okien czasowych oznacza, że na planie końcowym muszą się znaleźć wszystkie lekcje przewidziane w siatce zajęć dla danej grupy. Jest to podstawowe założenie bez którego układanie planu traci swój sens.
Często jednak dla konkretnej siatki zajęć spełnienie już tych warunków jest niemożliwe. Wynika to najczęściej z za małej liczby zasobów (za mało nauczycieli mogących prowadzić jeden przedmiot, za mało sal). Algorytmy szukają wtedy planu z najmniejszą liczbą konfliktów, a pozostałe przypadki uzupełniamy już ręcznie szukając jakichś kompromisów (dołożenie okna czasowego, zatrudnienie dodatkowego nauczyciela lub pomieszczenie dwóch mniejszych grup w jednej sali).

Ograniczenia miękkie to tak naprawdę życzenia użytkowników co do tego jak dany plan ma wyglądać. Wpływają one na końcową ocenę planu zajęć i mogą mieć określone wagi w zależności od ich istotności. Niektóre plany zajęć układane są pod uczniów (mała liczba okienek, zajęcia równomiernie rozłożone, brak bloków zajęć danego typu, odpowiednia przerwa między zajęciami), a niektóre pod prowadzących (zajęcia skumulowane w dwa dni tygodnia by umożliwić prace na innej uczelni). Wszystko zależy od osoby konfigurującej algorytm i definiującej funkcję oceny planu zajęć.

Funkcja oceny planu zajęć najczęściej polega na nakładaniu punktów karnych za złamanie określonych ograniczeń. Każde ograniczenie ma ustawioną własną wartość punktów karnych przy czym ograniczenia twarde powinny mieć dużo większą wagę od ograniczeń miękkich. Im więcej punktów karnych tym plan jest gorszy. To właśnie na podstawie funkcji oceny większość algorytmów wyszukuje optymalny plan zajęć.

\newpage
\section {Rozmiar problemu.}

By dobrze zrozumieć potrzebę używania algorytmów heurystycznych przy wyszukiwaniu optymalnego planu zajęć warto zobrazować skalę problemu. W tym celu przedstawiony zostanie przykład planu zajęć dla amerykańskiej szkoły średniego rozmiaru. W tym przykładzie, każde ze zdarzeń ma już przyporządkowane zasoby.

\begin{itemize}
	\item Okna czasowe - 40
	\item Nauczyciele - 27
	\item Sale - 30
	\item Grupy uczniów - 24
	\item Zdarzenia - 832
	\item Ograniczenia - 4
\end{itemize}

Mamy dostępne 40 okien czasowych, co rozłożone na 5 dni tygodnia daje 8 godzin zajęć dziennie. Jednocześnie odbywać się mogą 24 zajęcia. Liczba ta wynika z ograniczenia niepodzielności zasobów i jest minimalną liczbą z pośród ilości nauczycieli, sal i grup uczniów. Maksymalna liczba zdarzeń, które mogą się odbywać w danym tygodniu wynosi więc:
\[ 40 * 24 = 960 \]
\[ 960 > 832 \]

Dostępna liczba zdarzeń jest większa od liczby zdarzeń zawartych w siatce zajęć podanego przykładu, więc teoretycznie realizacja tego planu zajęć jest możliwa.

Na jedną grupę przypada około 35 zajęć, które muszą być rozłożone na 40 okien czasowych. Uwzględniając możliwość pojawienia się okienek w czasie zajęć mamy więc 40 wyrazowy ciąg zdarzeń czyli permutację. Liczba możliwości w jaki sposób można te zajęcia rozłożyć, uwzględniając możliwość okienek, wynosi więc:
\[40! = 815915283247897734345611269596115894272000000000\]

Jest to 48 cyfrowa liczba możliwości rozłożenia zajęć w oknach czasowych dla jednej grupy. Liczbę tę trzeba pomnożyć razy ilość grup.

Przedstawione dane mają z góry przyporządkowane zasoby do zdarzeń, więc każde zajęcia mają przyporządkowaną grupę, nauczyciela i salę. Dzięki temu obszar poszukiwań algorytmu zmniejsza się. Szukany jest tylko termin odbywania się zajęć. Tak przygotowane dane najtrafniej odzwierciedlają realia współczesnych szkół, gdzie najczęściej prowadzący z góry przypisany jest do jakiejś grupy, a konkretne przedmioty mogą się odbywać tylko w przystosowanych do tego celu salach.

Takie podejście, w którym zasoby z góry są przypisane do zdarzeń, umożliwia szybszą pracę algorytmu. Jednak trzeba pamiętać, że ma to również wpływ na wynik końcowy. Może się zdarzyć przypadek, w który zamiana nauczycieli sprawi, że cały plan znacznie się zmieni i osiągniemy o wiele lepszy rezultat. Jednak z uwagi na złożoność obliczeniową takiego rozwiązania, jak i na dostępną liczbę danych testowych, które mają już przypisane zasoby, zdecydowano się wybrać wariant, w którym algorytm poszukuje tylko okien czasowych dla wszystkich dostępnych zdarzeń.