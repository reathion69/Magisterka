\chapter{Podsumowanie}

Głównym celem niniejszej pracy było przystosowania algorytmu przeszukiwania z Tabu do problemu układania planu zajęć. Cel ten został zrealizowany. Efektem pracy jest aplikacja potrafiąca ułożyć plan zajęć dla problemu zapisanego w formacie XHSTT. W aplikacji zaimplementowano cztery ograniczenia planu zajęć spośród dużej liczby dostępnych ograniczeń w formacie XHSTT. Pozostałe ograniczenia nie zostały zaimplementowane z powodu rzadkiego występowania w różnych danych testowych oraz ograniczeń czasowych spowodowanych terminem oddawania pracy. 

Możliwy jest dalszy rozwój aplikacji w postaci implementacji kolejnych ograniczeń oraz implementacji poprawnego zachowania algorytmu dla zajęć o długości dłuższej niż trzy okna czasowe. Osiągnięte rezultaty są satysfakcjonujące dla autora, jednak można starać się jeszcze bardziej zoptymalizować algorytm poprzez dodanie metod dywersyfikacji i intensyfikacji do algorytmu przeszukiwania z Tabu. Należy się jednak liczyć z tym, że każde dodatkowe ograniczenie wydłuży czas działania algorytmu, konieczna więc będzie jego optymalizacja.

Podczas implementacji algorytmu, autor napotkał wiele problemów. Wszystkie zostały w pełni rozwiązane. Pierwszym z nich było zrozumienie formatu danych wejściowych XHSTT, który, mimo bardzo dobrej dokumentacji, jest formatem dość skomplikowanym i zmusza użytkownika do spojrzenia na problem rozwiązywania planu zajęć z innej strony. Kolejnym problemem była sama implementacja algorytmu ze względu na dane wejściowe. Rozmiar danych wejściowych był bardzo duży i trudno było zweryfikować poprawność pracy algorytmu. Dlatego zdecydowano się rozpocząć implementację, testując jej działanie na danych nierzeczywistych zawierających bardzo małą liczbę zdarzeń i zasobów. Podczas testowania pojawił się problem długiego oczekiwania na wyniki oraz problem obiektywnego porównania rozwiązań. Zdecydowano się generować raporty z pracy algorytmu w celu analizy jego zachowań w zależności od parametrów wejściowych.

Aspektem badawczym pracy była również analiza pracy algorytmu dla różnych wartości parametru długości Tabu. Dzięki odpowiedniej analizie pokazano, że najlepsze rezultaty algorytm osiągał dla listy Tabu długości 4-5 \% liczby wszystkich dostępnych ruchów. Długość pracy algorytmu zależna jest od rozmiaru danych wejściowych oraz od wartości parametrów. Dla problemu układania planu zajęć dla szkoły rozmiaru statystycznego polskiego liceum oraz optymalnie dobranych parametrów wejściowych, czas pracy algorytmu wynosi około 10 minut.

Zaimplementowany algorytm nadaje się do użytku codziennego i może być wykorzystany dla większości problemów układania planu zajęć, zapisanych w formacie XHSTT.

