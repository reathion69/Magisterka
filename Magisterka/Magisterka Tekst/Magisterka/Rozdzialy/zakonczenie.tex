\chapter{Podsumowanie}

Głównym celem niniejszej pracy było przystosowania algorytmu przeszukiwania z Tabu do problemu układania planu zajęć. Cel ten został zrealizowany. Effektem pracy jest aplikacja potrafiąca ułozyć plan zajęć dla problemu zamisanego w formacie XHSTT. W aplikacji zaimplementowano cztery ograniczenia planu zajęć spośród dużej liczby dostępnych ograniczeń w formacie XHSTT. Pozostałe ograniczenia nie zostały zaimplementowane z powodu rzadkiego występowania w różnych danych testowych oraz ograniczeń czasowych spowodowanych terminem oddawania pracy. 

Możliwy jest dalszy rozwój aplikacji w postaci implementacji kolejnych ograniczeń oraz implementacji poprawnego zachowania algorytmu dla zajęć o długości dłuższej niż trzy okna czasowe. Osiągnięte rezultaty są satysfakcjonujące dla autora, jednak można starać się jeszcze bardziej zoptymalizować algorytm poprzez dodanie metod dywersyfikacji i intensyfikacji do algorytmu przeszukiwania z Tabu.

Podczas implementacji algorytmu autor napotkał wiele problemów, które jednak zostały w pełni rozwiązane. Pierwszym z nich było zrozumienie formatu danych wejściowych XHSTT, który, mimo bardzo dobrej dokumentacji, jest formatem dość skomplikowanym i zmusza użytkownika do spojrzenia na problem rozwiązywania planu zajęć z innej strony. Kolejnym problemem była sama implementacja algorytmu pod kątem danych wejściowych. Rozmiar danych wejściowych był bardzo duży i trudno było zweryfikować poprawność pracy algorytmu. Dlatego zdecydowano się rozpocząć implementację, testując jej działanie na danych nierzeczywistych zawierającyh bardzo małą liczbę zdarzeń i zasobów.