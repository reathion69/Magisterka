\chapter{Podsumowanie}

Głównym celem niniejszej pracy było zastosowanie algorytmu przeszukiwania z tabu do rozwiązywania problemu układania planu zajęć. Cel ten został zrealizowany. Efektem pracy jest aplikacja, za pomocą której można ułożyć plan zajęć dla sformułowania problemu zapisanego w formacie XHSTT. W aplikacji zaimplementowano cztery ograniczenia planu zajęć spośród dużej liczby dostępnych ograniczeń w formacie XHSTT. Pozostałe ograniczenia nie zostały zaimplementowane z powodu rzadkiego występowania w różnych danych testowych oraz ograniczeń czasowych wykonania niniejszej pracy. 

Podczas implementacji algorytmu, autor napotkał wiele trudności. Wszystkie z nich zostały pokonane. Pierwszą z nich było zapoznanie się z formatem danych wejściowych XHSTT, który mimo bardzo dobrej dokumentacji jest formatem stosunkowo złożonym i sprawia, że na problem rozwiązywania planu zajęć należy spojrzeć z innej strony. Kolejną trudnością była implementacja algorytmu ze względu na rozmiar danych wejściowych. Rozmiar ten był bardzo duży i trudno było zweryfikować poprawność pracy algorytmu. Dlatego zdecydowano się rozpocząć implementację, testując jej działanie na danych nierzeczywistych zawierających bardzo małą liczbę zdarzeń i zasobów. Podczas testowania pojawiła się trudność związana z długim czasem oczekiwania na wyniki oraz obiektywnym porównaniem jakości rozwiązań. Zdecydowano, aby generować raporty z pracy algorytmu w celu analizy jego działania w zależności od parametrów wejściowych.

Aspektem badawczym pracy była analiza pracy algorytmu dla różnych wartości długości tabu. W wyniku przeprowadzonej analizy pokazano, że najlepsze rezultaty algorytm osiągał dla listy tabu długości 4-5\% liczby wszystkich dostępnych ruchów. Czas pracy algorytmu zależny jest od rozmiaru danych wejściowych oraz od przyjętych wartości parametrów. Dla problemu układania planu zajęć dla szkoły o rozmiarze polskiego liceum o średniej wielkości oraz optymalnie dobranych parametrach, czas pracy algorytmu wynosi około 10 minut.

Zaimplementowany algorytm nadaje się do praktycznego użytkowania i może być wykorzystany dla większości problemów układania planu zajęć zapisanych w formacie XHSTT.

Możliwy jest dalszy rozwój aplikacji w postaci implementacji kolejnych ograniczeń oraz implementacji poprawnego zachowania algorytmu dla zajęć o długości dłuższej niż trzy okna czasowe. Osiągnięte rezultaty są satysfakcjonujące, jednak można poprawić starania, aby poprawić algorytm przeszukiwania przez dodanie metod dywersyfikacji i intensyfikacji do opracowanego algorytmu. Należy się jednak liczyć z tym, że każde dodatkowe ograniczenie wydłuży czas działania algorytmu, konieczna więc będzie jego optymalizacja.
