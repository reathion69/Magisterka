\chapter{Wstęp}


Problem układania planu zajęć jest dobrze znany w środowiskach szkolnych i akademickich. Układanie planu to zajęcie żmudne oraz wymagające dużej spostrzegawczości i wyobraźni u osoby, która plan układa. Problem ten jest aktualny każdego roku, lub gdy okoliczności sprzyjają, przy zmianie liczby prowadzących, bądź zmienionej siatce zajęć. Układanie planu zajęć metodami tradycyjnymi pochłania dużo czasu i często kończy się niepowodzeniem. Jeżeli już uda się ułożyć plan, to często nie jest on najlepszy i nie spełnia oczekiwań uczniów bądź nauczycieli. Dlatego też badacze w wielu środowiskach na świecie rozważają problem układania planu zajęć i poszukują efektywnych algorytmów jego rozwiązania. W obecnych czasach mało która szkoła lub uczelnia nie wspomaga się przy tej czynności komputerem. Możliwość szybkiego utworzenia i ocenienia wielu planów lub uzyskanie wiedzy, czy dany plan jest możliwy do zrealizowania, to tylko niektóre zalety rozwiązania tego problemu za pomocą komputera. Najważniejszą zaletą jest możliwość skorzystania z algorytmów, które w krótkim czasie, np. 10 minut, ułożą plan zajęć, zgodnie z podanymi założeniami.

Celem niniejszej pracy jest zbadanie możliwości zastosowania algorytmu przeszukiwania z tabu do problemu układania planu zajęć. Algorytm ten, rozwiązujący trudne problemy z wielu dziedzin życia, może mieć również zastosowany do rozwiązywania problemu układania planu zajęć. Praca oparta jest na materiałach źródłowych, w których do optymalizacji układania planu zajęć wykorzystano różne algorytmy heurystyczne. Korzystając z doświadczeń poprzedników, udało się uniknąć typowych błędów pojawiających się przy implementacji algorytmów rozwiązywania problemu układaniu planu zajęć.

Aspektem badawczym pracy jest zbadanie wpływu parametrów zaimplementowanego algorytmu na szybkość jego działania i otrzymane wyniki. Korzystając dużej bazy danych testowych pochodzących z różnych szkół i uczelni świata dokonamy testowania algorytmu dla rzeczywistych problemów układania planu zajęć.

Praca, poza wstępem i zakończeniem, składa się z pięciu rozdziałów. Pierwszy rozdział poświęcony jest omawianiu algorytmu przeszukiwania z tabu. Przedstawiono w nim cechy algorytmów heurystycznych i metaheurystycznych oraz genezę algorytmu przeszukiwania z Tabu. W rozdziale znaleźć można również wyjaśnienie podstawowej idei działania algorytmu oraz różnych jego wariantów. Zdefiniowano także podstawowe pojęcia związane z algorytmem, takie jak sąsiedztwo, kryterium aspiracji, dywersyfikacja, intensyfikacja etc. Drugi rozdział formułuje problem układania planu zajęć. zawiera formalizację problemu i omówienie ograniczeń związanych z planami zajęć. Przedstawiona jest również złożoność problemu układania planu zajęć, w celu uzasadnienia potrzeby zastosowania algorytmów heurystycznych. Kolejny rozdział w całości poświęcony jest formatowi danych wejściowych, jakie są niezbędne do zaimplementowania algorytmu w celu poprawnego działania. Wykorzystany format danych jest formatem wykorzystywanym w problemie układania planu zajęć przez wielu badaczy na całym świecie. Opisanie formatu pozwala zrozumieć problemy, z jakimi trzeba się zmierzyć podczas implementowania algorytmu. Implementacja algorytmu oraz jego specyfikacja wewnętrzna i zewnętrzna zostały opisane w następnym rozdziale. Przedstawiono w nim funkcję wykorzystane w jej realizacji. Omówiono również wykorzystane podczas pracy narzędzia i środowiska. W ramach specyfikacji zewnętrznej przedstawiono opis interfejsu graficznego użytkownika oraz sposób korzystania z aplikacji. Ostatni rozdział opisuje proces testowania algorytmu. Omówiona została w nim specyfikacja techniczna środowiska testowego oraz zestaw danych wybrany do testowania. Wyniki działania aplikacji zostały zaprezentowane w postaci dwóch wygenerowanych planów zajęć wraz z oceną uzyskanych rozwiązań. Na końcu rozdziału zaprezentowano wpływ parametru długości listy tabu na wyniki działania algorytmu.