\chapter{Wprowadzenie}


Problem układania planu zajęć jest problem szeroko znanym w kręgach szkolnych i akademickich. Jest to zajęcie żmudne oraz wymagające dużej spostrzegawczości i wyobraźni u osoby układającej. Problem ten powraca każdego roku lub, gdy okoliczności sprzyjają, przy zmianie liczby prowadzących, bądz zmienionej siatce zajęć. Układanie planu zajęć metodami tradycyjnymi pochłania bardzo dużo czasu i często kończy się niepowodzeniem. Jeżeli już uda się ułożyć plan, to często jest on nieoptymalny i nie spełnia oczekiwań uczniów bądz nauczycieli. Dlatego też naukowcy na całym świecie badają problem układania planu zajęć i poszukują rozwiązań optymalizujących to zagadnienie. Dzisiaj, mało która szkoła lub uczelnia nie wspomaga się przy tej czynności komputerem. Możliwość szybkiego przetestowania i ocenienia wielu możliwości lub natychmiastowa wiedza, czy dany plan jest możliwy do zrealizowania, to tylko niektóre zalety przeniesienia tego problemu do środowiska cyfrowego. Najważniejszą zaletą jest możliwość ułożenia algorytmów, które w czasie 10 minut ułożą dla nas plan zajęć, według podanych przez nas wytycznych.

Celem tej pracy jest zbadanie możliwości przystosowania algorytmu przeszukiwania z Tabu do problemu układania planu zajęć. Algorytm ten, jako algorytm optymalizujący rozwiązania z wielu dziedzin życia, powinien znaleźć swoje zastosowanie również przy optymalizacji rozwiązania problemu układania planu zajęć. Praca ta, w dużej mierze oparta jest na innych pracach dyplomowych, w których do optymalizacji układania planu zajęć wykorzystano inne algorytmy przeszukiwania. Bazując na doświadczeniach poprzedników, udało się uniknąć wielu typowych błędów pojawiających się przy implementacji algorytmów przeszukiwania, czy przy układaniu planu zajęć.

Aspektem badawczym pracy, oprócz przystosowania algorytmu do układania planu zajęć, jest również zbadanie wpływu parametru algorytmu, określanego jako długość Tabu, na pracę algorytmu i otrzymane wyniki. Dzięki dużej bazie danych testowych, pochodzących z różnych szkół i uczelni świata, można przetestować algorytm dla rzeczywistych problemów układania planu zajęć.

Praca ta podzielona jest, poza wstępem i zakończeniem, na pięć rozdziałów opisujących cały zakres prac badawczych jaki autor wykonał w celu zrealizowania założeń pracy magisterskiej. 

Pierwszy rozdział poświęcony jest algorytmowi wyszukiwania, bądź w zależności od nazewnictwa, algorytmowi przeszukiwania z Tabu. Omówione w nim są założenia algorytmów heurystycznych i metaheurystycznych oraz geneza samego algorytmu przeszukiwania z Tabu. W rozdziale znaleźć można również wyjaśnienie idei tego algorytmu, różnych jego wariantów i przekształceń. Opisane też są podstawowe pojęcia związane z algorytmem takie jak sąsiedztwo, kryterium aspiracji, dywersyfikacja czy intensyfikacji.

Drugi rozdział przybliża czytelnikowi problem układania planu zajęć. Można w nim znaleźć sposób formalizacji problemu i omówienie takich zagadnień jak ograniczenia planu zajęć. Przedstawiona jest tu również skala problemu układania planu zajęć, w celu lepszego zrozumienia potrzeby zastosowania algorytmów heurystycznych.

Kolejny rozdział w całości poświęcony jest formatowi danych wejściowych, jakie potrzebuje zaimplementowany algorytm w celu poprawnego działania. Wykorzystany format danych jest formatem wykorzystywanym w problemie układania planu zajęć przez wielu naukowców na cąłym świecie, dlatego jego opisanie pozwala czytelnikowi lepiej zrozumieć problemy z jakimi trzeba się zmierzyć podczas implementacji algorytmu, który ma za zadanie rozwiązać problem układania planu zajęć.

Implementacja algorytmu oraz specyfikacja wewnętrzna i zewnętrzna zostały opisane w nastepnym rozdziale. Przybliżona jest w nim cała idea implementacji algorytmu oraz przedstawione są funkcję wykorzystane w jej realizacji. Przedstawiono tu również wykorzystane podczas pracy technologie. W specyfikacji zewnętrznej można znaleźć opis interfejsu graficznego użytkownika i dowiedzieć się w jaki sposób korzystać z aplikacji.

Ostatni rozdział opisuje proces testowania algorytmu. Omówiona została w nim specyfikacja techniczna środowiska testowego oraz zestaw danych wybrany do testowania. Wyniki działania aplikacji zostały zaprezentowane w postaci dwóch wygenerowanych planów zajęć oraz ogólnej oceny wygenerowanego rozwiązania. Na końcu rozdziału zaprezentowano wpływ parametru długości Tabu na działanie algorytmu.