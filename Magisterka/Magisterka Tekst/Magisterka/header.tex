%%%%%%%%%%%%%%%%%%%%%%%%%%%%%%%%%%%%%%%%%%%%%%%%%%%%%%%%%%%%%%%%%%%%%%%%%%%
% This is a sample header for a sample dissertation. Fill in the name,
% and the other information. LaTeX will work out the table of
% content, the list of figures and of tables for you.
%%%%%%%%%%%%%%%%%%%%%%%%%%%%%%%%%%%%%%%%%%%%%%%%%%%%%%%%%%%%%%%%%%%%%%%%%%%

\newpage
\thispagestyle{empty}




% ******* Title page *******
% **************************

\begin{onehalfspacing}
\begin{center}

\centering
\includegraphics[keepaspectratio,scale=0.1]{./figures/godlo.PNG} \\[.8cm]


{\fontsize{17}{17}\selectfont
\textsc{Politechnika Śląska \\[.3cm]
Wydział Automatyki, Elektroniki i Informatyki  \\[.3cm]
Kierunek Informatyka  \\[2.5cm]}
\textbf{Praca dyplomowa magisterska \\[1.7cm]}}



\large 
{Algorytm wyszukiwania z tabu do rozwiązywania problemu układania planu zajęć} \\[3.1cm]
% Jeśli tytuł pracy zajmuje 2 linijki, wartość [2.3cm] zamieniamy na [3.1cm], jeśli tylko jedną - na [3.9cm] i odwrotnie - zwiększając liczbę linijek o jedną (do czterech) zmieniamy na [1.5cm] itd.


\large
\begin{flushleft}
Autor: Michał Szluzy  \\
Kierujący pracą:  prof. dr hab. inż. Zbigniew Czech \\
\end{flushleft}

\vspace{2cm}
Gliwice, czerwiec 2017
\end{center}
\end{onehalfspacing}

\singlespacing
\newpage

\thispagestyle{empty}
\mbox{}


%ABSTRACT
\begin{abstract}
W niniejszej pracy został opisany proces przystosowania algorytmu wyszukiwania z tabu do rozwiązywania problemu układania planu zajęć. Aspektem badawczym pracy było zbadanie wpływu parametrów zaimplementowanego algorytmu na szybkość jego działania i otrzymane wyniki. Omówione zostały podstawy teoretyczne badanego problemu oraz trudności towarzyszące implementacji algorytmu jego rozwiązania. Testowanie algorytmu zostało przeprowadzone na danych rzeczywistych, a wyniki zaprezentowane na wykresach.
\end{abstract}
%END OF ABSTRACT

\setcounter{page}{4} 
\doublespacing
\mbox{}

%\pagestyle{empty}
%\pagenumbering{Roman}
%\setcounter{page}{0} \pagestyle{plain}


\tableofcontents

\listoffigures
%\listoftables

\newpage
\thispagestyle{empty}


%\pagestyle{fancy}